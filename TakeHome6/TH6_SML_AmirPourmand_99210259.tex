\documentclass{article}[12pt]
\usepackage{graphicx}
\usepackage{amsmath,amssymb}
\usepackage{tikz}
\usepackage{xepersian}
\settextfont[Scale=1]{IRXLotus}
\setlatintextfont[Scale=0.8]{Times New Roman}

\DeclareRobustCommand{\bbone}{\text{\usefont{U}{bbold}{m}{n}1}}

\DeclareMathOperator{\EX}{\mathbb{E}}% expected value
%\DeclareMathOperator{\Pr}{\mathbb{P}}

\title{  \includegraphics[scale=0.35]{../../Images/logo.png} \\
    دانشکده مهندسی کامپیوتر
    \\
    دانشگاه صنعتی شریف
}
\author{استاد درس: دکتر حمیدرضا ربیعی}
\date{بهار ۱۴۰۰}



\def \Subject {
تمرین در خانه ششم
}
\def \Course {
درس یادگیری ماشین آماری
}
\def \Author {
نام و نام خانوادگی:
امیر پورمند}
\def \Email {\lr{pourmand1376@gmail.com}}
\def \StudentNumber {99210259}


\begin{document}

 \maketitle
 
\begin{center}
\vspace{.4cm}
{\bf {\huge \Subject}}\\
{\bf \Large \Course}
\vspace{.8cm}

{\bf \Author}

\vspace{0.3cm}

{\bf شماره دانشجویی: \StudentNumber}

\vspace{0.3cm}

آدرس ایمیل
:
{\bf \Email}
\end{center}


\clearpage
\section{سوال ۱}

\begin{enumerate}
\item 
در واقع مدل ما باید سه ویژگی داشته باشد:
\lr{global} ,
\lr{model-specific},
\lr{decision tree}.
\lr{global}
زیرا که باید کل مدل را یکجا به تیم پزشکی توضیح داد تا دلایل را متوجه شوند و یکی از بهترین روش های توضیح مدل که برای پزشکی نیز قابل قبول باید درخت تصمیم است که خود آنها نیز استفاده میکنند و مشخصا این روش یک روش مبتنی بر مدل است. 
\item
برای این مدل نیز سه ویژگی متصور است:
\lr{local},
\lr{model-agnostic},
\lr{LIME}.
local
باشد زیرا که نیاز نیست کل مدل یکجا درک شود و همین که یک مثال خاص برایش توضیح باشد کافیست و البته میتواند model-agnostic باشد زیرا بتوانیم مدل های مختلف را اپلای کنیم و مشکلی نباشد. با این تعاریف مدل لایم یک مدل خوب است زیرا هم model-agnostic است و هم local عمل میکند که برای مدل های black-box هم مناسب است.  
\end{enumerate}

\clearpage
\section{سوال ۲}
\subsection{رگرسیون خطی یا رگرسیون لاجیستیک}
\subsubsection{simulatibilty}
عوامل تصمیم گیرنده را انسان نیز میتواند مشخص کند و مشکلی نیست و البته تعامل یا اینتراکشن بین انها نیز کمترین مقدار است. 
\subsubsection{decomposability}
متغیرها مشخص هستند اما تعامل بین آنها بیشتر شده تا مدل بیشتر decomposable باشد. 
\subsubsection{\lr{algorithmic transparency}}
بدون روش های آماری پیشرفته نمیتواند متغیرها و تاثیرات آنها را به سادگی بررسی کرد.
\subsubsection{post-hoc}
نیاز نیست. 
\subsection{درخت تصمیم}
\subsubsection{simulatibilty}
یک انسان بدون 
هیچ دانش ریاضی ای هم میتواند درخت را شبیه سازی و آزمایش کند و مشکلی ندارد. 
\subsubsection{decomposability}
مدل تفکیک پذیر هست زیرا از یک سری قانون تشکیل شده که خیلی مشخص است و هر کسی میتواند آنها را به خودی خود درک کند.
\subsubsection{\lr{algorithmic transparency}}
شفاف است زیرا قواعد آن توسط انسان قابل درک و فهم است و به سادگی میتوان فرآیند را درک کرد. 
\subsubsection{post-hoc}
نیازی نیست. 
\subsection{KNN}
\subsubsection{simulatibilty}
قابل انجام است زیرا پیچیدگی خاصی ندارد و انسان نیز میتواند انجام دهد.
\subsubsection{decomposability}
اگرچه تعداد متغیرها خیلی زیاد است و به سادگی مدل درخت تصمیم نیست ولی میتوان آنرا به صورت تک به تک آنالیز کرد. 
\subsubsection{\lr{algorithmic transparency}}
قوانین به قدری پیچیده هستند که برای فهم مدل به ابزار ریاضی مورد نیاز است. 
\subsubsection{post-hoc}
نیازی نیست. 
\subsection{\lr{Bayesian Models}}
\subsubsection{simulatibilty}
مشکلی ندارد و میتوان روابط آماری را توسط افراد مورد بررسی قرار داد و از این نظر واضح است. 
\subsubsection{decomposability}
روابط آماری خیلی پیچیده شده اند اما میتوان با تجزیه آنها به حالت مارجینال مدل را decomposable کرد. 
\subsubsection{\lr{algorithmic transparency}}
حتی اگر مدل را بتوانیم decompose کنیم. بررسی شفافیت مدل به این سادگی ها نیست و فقط با ابزار ریاضی میتوان آنرا انجام داد. 
\subsubsection{post-hoc}
نیازی نیست. 
\subsection{\lr{Tree ensemble}}
\subsubsection{simulatibilty}
ندارد
\subsubsection{decomposability}
ندارد
\subsubsection{\lr{algorithmic transparency}}
ندارد
\subsubsection{post-hoc}

نیاز دارد و معمولا از روش ساده سازی مدل یا استخراج اهمیت فیچرها استفاده میشود.

\subsection{\lr{Support Vector Machine}}

\subsubsection{simulatibilty}
ندارد
\subsubsection{decomposability}
ندارد
\subsubsection{\lr{algorithmic transparency}}
ندارد
\subsubsection{post-hoc}
معمولا از ساده سازی مدل یا توضیحات محلی استفاده میشود. 
\subsection{\lr{Multi-Layer Network}}
\subsubsection{simulatibilty}
ندارد
\subsubsection{decomposability}
ندارد

\subsubsection{\lr{algorithmic transparency}}
ندارد
\subsubsection{post-hoc}
نیاز دارد و معمولا از روش ساده سازی مدل، اهمیت ویژگیها و یا مصور سازی استفاده میشود

\subsection{\lr{Convolutional Neural Network}}
\subsubsection{simulatibilty}
ندارد
\subsubsection{decomposability}
ندارد
\subsubsection{\lr{algorithmic transparency}}
ندارد
\subsubsection{post-hoc}
معمولا از روش های مبتنی بر اهمیت ویژگی یا مثلا مصور سازی داده استفاده میشود

\subsection{\lr{Recurrent Neural Network}}
\subsubsection{simulatibilty}
ندارد
\subsubsection{decomposability}
ندارد
\subsubsection{\lr{algorithmic transparency}}
ندارد
\subsubsection{post-hoc}
معمولا از روش های اهمیت ویژگی استفاده میشود. 
\end{document}