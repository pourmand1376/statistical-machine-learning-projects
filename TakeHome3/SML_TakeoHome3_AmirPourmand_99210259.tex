\documentclass{article}[12pt]
\usepackage{amsmath,amssymb}
\usepackage{xepersian}
\settextfont{IRXLotus}
\setlatintextfont[Scale=1]{Times New Roman}

\DeclareRobustCommand{\bbone}{\text{\usefont{U}{bbold}{m}{n}1}}

\DeclareMathOperator{\EX}{\mathbb{E}}% expected value


\title{   
    دانشکده مهندسی کامپیوتر
    \\
    دانشگاه صنعتی شریف
}
\author{استاد درس: دکتر حمیدرضا ربیعی}
\date{بهار ۱۴۰۰}



\def \Subject {
تمرین در خانه سوم
}
\def \Course {
درس یادگیری ماشین آماری
}
\def \Author {امیر پورمند}
\def \Report {سری سوم تمرین ها}
\def \StudentNumber {99210259}


\begin{document}

 \maketitle
 
\begin{center}
\vspace{.4cm}
{\bf {\huge \Subject}}\\
{\bf \Large \Course}
\vspace{.2cm}
\end{center}
{\bf \Author }  \\
{\bf شماره دانشجویی:\ \StudentNumber}
\hspace{\fill} 
{\Large \Report} \\
\hrule
\vspace{0.8cm}

\clearpage
 
\section{سوال ۱}
\subsection{a}
پس با توجه به جمله اول سوال در واقع 
$N(t) =1 $
می باشد. 
باید توزیع زمان رخداد رو بدست بیاریم.


در ابتدا در نظر میگیریم که متغیر 
تصادفی 
$s$
در واقع همان زمان رخ‌دادن ماست. 

مثل همیشه برای بدست آوردن توزیع ابتدا CDF 
آن را محاسبه می‌کنیم. 
بنابراین داریم
\begin{equation}
\begin{split}
F_{S|N(t)=0}(s) &= P(S \leq s | N_t = 0)\\
 &= 
P(N(s) = 1 | N(t) = 1)  \\ &= 
\frac{P(N(s)=1, N(t) = 1)}{P(N(t)=1)} \\& = 
\frac{P(N(s)=1,N(t)-N(s)=0)}{P(N(t)=1)}
\\&= 
\frac{\exp(-\lambda s)(\lambda s)^1 
\exp(- \lambda (t-s)) ((\lambda(t-s))^0/(0!))
}{
\exp(- \lambda t)(\lambda t)^1 / (1!)
}
  \\&=
 \frac{
exp(- \lambda t ) \lambda s }
{
exp(- \lambda t) (\lambda t) 
} \\&= \frac{s}{t}
\end{split}
\end{equation}
بنابراین یک توزیع یکنواخت داریم. 
البته دقت داشته باشیم متغیر ما در اینجا 
$s$
است یعنی CDF  در واقع به این صورت است. 

\begin{equation*}
\frac{s-0}{t-0}
\end{equation*}
که این بدین معناست که در بازه ۰ تا 
$t$
متغیر ما توزیع یکنواخت دارد 
\subsection{b}

خب اول کمی mapping انجام دهیم. 
میدانیم دو تا اتفاق افتاده که آن ها را با 
$s_1,s_2$
نشان می دهیم. و مشخصا
میدانیم که 
$0 \leq S_1 \leq S_2 \leq 3$
است

خب حالا باید ابتدا توزیع شرطی s1 به شرط 
$N(t)=2$
را بدست آوریم که برابر است با

\begin{equation}
\begin{split}
F_{S1|N(t)=2}(s1) &=
\frac{
P(S_1 \leq s1 , N(t) = 2) }
{
P(N(t)=2)
} \\& = 
\frac{
P( N(s1) \geq 1 ,N(t) = 2)}
{
P(N(t)=2)
}
\\& =\frac {P(N(s_1)=1,N(t)=2)+P(N(s_1)=2,N(t)=2)} {P(N(t)=2)}
\\& = 
\frac{
P(N(s_1)=1, N(t)-N(s_1)=1) + P(N(s_1)=2,N(t)-N(s1)=0)}
{P(N(t)=2)}
\\& = 
\frac{e^{-\lambda s_1 }(\lambda s_1)
e^{-\lambda (t-s_1)} \lambda(t-s_1)
+ e^{-\lambda s_1} (\lambda s_1)^2 / 2! * 
e^{-\lambda (t-s_1)}
 }
{
e^{-\lambda t}(\lambda t)^2 /(2!)
} 
\\ &=
\frac{
 \lambda^2 e^{- \lambda t} s_1 (t-s1) + 
 e^{-\lambda t} (\lambda s_1) / 2
}{
\lambda ^2 t^2 e^{-\lambda t} /2
} 
\\
&=
\frac{2 s_1 t - s_1^2}{t^2}
\end{split}
\end{equation}

پس داریم توزیع متغیر s1 به این صورت خواهد بود و از مشتق CDF نسبت به s1 بدست خواهد آمد.

\begin{equation}
\begin{split}
f_{S1|N(t)=2}(s1) &= 
\frac{\partial F}{\partial s_1}\\&= 
\frac{\partial 2s_1 t - s_1^2 }{t^2 \partial s_1} \\& =
\frac{\partial 2 s_1 t }{t^2 \partial s_1} - 
\frac{\partial s_1^2 }{t^2 \partial s_1} \\& 
= \frac{2t}{t^2} - \frac{2s_1}{t^2} \\& 
= \frac{2(t-s_1)}{t^2}
\end{split}
\end{equation}

حال باید امیدریاضی توزیع شرطی متغیر s
را بدست اوریم که میشود

\begin{equation}
\begin{split}
\EX(s_1 | N(t)=2) = \int_{0}^{t}
s_1 \frac{2(t-s_1)}{t^2} ds_1= \frac{t}{3}
\end{split}
\end{equation}

حال به قسمت دوم سوال میرسیم و باید در اینجا توزیع 
$s_2$
را بدست بیاوریم. 
باز هم به سراغ CDF رفته 
و بعد از آن از آن مشتق میگیریم تا توزیع شرطی بدست اید. 
خب شروع میکنیم


\begin{equation}
\begin{split}
F_{S_2|N(t)=2}(s_2) &= 
\frac{
P(S_2 \leq s_2 , N(t) = 2)}
{
P(N(t)=2)
}
\\& = 
\frac{P(N(s_2)=2, N(t)=2}{P(N(t)=2}
\\& = 
\frac{P(N(s_2)=2 , N(t)-N(s_2)=0)}{P(N(t)=2)}
\\& = 
\frac{
\frac{ e^{-\lambda s_2 }(\lambda s_2)^2}{(2!)} 
e^{-\lambda (t-s_2)} 
}
{
\frac{
e ^ {-\lambda t} (\lambda t)^2 / (2!)
} {2!}
} \\ &= 
\frac{\lambda ^2 s_2^2 }{\lambda ^2 t^2} \\
& = \frac{s_2^2}{t^2}
\end{split}
\end{equation}

حال باید طبق روال قبل توزیع شرطی این متغیر را بدست آوریم که برابر مقدار زیر خواهد بود

\begin{equation}
f_{S2|N(t)=2}(s_2) = \frac{\partial s_2^2}{t^2 \partial s_2} = 
\frac{2 s_2}{t^2}
\end{equation}
و امیدریاضی 
شرطی متغیر 
$s_2$
برابر مقدار زیر خواهد شد
\begin{equation}
\EX[s_2 | N(t)=2] = 
\int_0^t s_2 \frac{2 s_2 }{t^2} ds_2= 
\frac{2}{t^2} \frac{t^3}{3} = \frac{2t}{3}
\end{equation}
\section{سوال ۲
}
\subsection{توزیع T1}

خب ابتدا باید توزیع تجمعی 
$T_1$ 
و بعد 
$T_2$
را بدست آوریم و بعد مشتق بگیریم که توزیع بدست آید. 
برای 
$T_1$
داریم:


\begin{equation}
\begin{split}
F_{T_1}(t) = P(T_1 \leq t) &= 1-P(T_1 \geq t)
\\
&= 1- P(N(t) = 0) 
\\
&= 
1- exp(-\int_0^t \frac{1}{1+k}dk) 
\\
&= 1- exp(-(ln(t+1)-ln(0+1))
\\
&= 1- exp(ln(\frac{1}{1+t}))
\\
&= 1 - \frac{1}{1+t}
\end{split}
\end{equation}

حال تابع توزیع از مشتق نسبت به t بدست می آید. 

\begin{equation}
\begin{split}
f_{T_1}(t) &=
\frac{\partial}{\partial t} F_{T_1}(t) \\ &=
\frac{\partial}{\partial t} \frac{t}{t+1} 
\\
&= 
\frac{1(t+1) - 1(t)}{(1+t)^2}
\\
&= \frac{1}{(1+t)^2}
\end{split}
\end{equation}


\subsection{توزیع T2}
حال باید توزیع متغیر 
$T_2$
را بدست آوریم. 

\begin{equation}
\begin{split}
P(T_2 \geq t) &= P(N(t) = 1) + P(N(t)=0)\\
&=
exp(-\int_0^t \frac{1}{1+k}dk) (\int_0^t \frac{1}{1+k}dk) +exp(-\int_0^t \frac{1}{1+k}dk)\\
&= 
exp(-log(1+t))(log(1+t)) +exp(-log(1+t))\\
&= 
\frac{log(1+t)}{1+t} + \frac{1}{1+t}
\end{split}
\end{equation}

بنابراین داریم

\begin{equation}
F_{T_2}(t) = P(T_2 \leq t) = 1- \frac{log(1+t)}{1+t} - \frac{1}{1+t} 
\end{equation}

که اگر از ان مشتق بگیریم داریم:

\begin{equation}
f_{T2}(t) = \frac{\partial}{\partial t} F_{T_2}(t) = \frac{ln(t+1)}{(t+1)^2}
\end{equation}

البته این زمان رسیدن است و چیزی که در سوال خواسته شده interarrival تایم هست که آن را در ادامه بدست می اورم. یعنی تفاوت در این است که الان 
$T_1$
در واقع 
زمان رسیدن است و 
در زیر 
$T_1$ 
زمان بین دو اتفاق است که در اینصورت جواب سوال در قسمت زیر می آید. 

ابتدا میدانیم  
$N(T_1+t) - N(T_1)$
از توزیع پواسون پیروی میکند و نرخ آن برابر خواهد بود با:

\begin{equation}
\int_{T_1}^{T_1+t} \frac{1}{1+k}dk = ln(1+T_1+t)-ln(1+T_1) = ln\frac{1+T_1+t}{1+T_1} 
\end{equation}

حال توزیع شرطی T2 به شرط T1
برابر خواهد بود با:

\begin{equation}
\begin{split}
P(T_2 > t | T_1) &= P(N(T_1 + t)-N(T_1) =0|T_1) \\
&= exp(- ln\frac{1+T_1+t}{1+T_1}) \\
&= 
\frac{T_1+1}{T_1+t+1}
\end{split}
\end{equation}

حالا توزیع 
$T_1$ را
داریم میتوانیم با ضرب کردن توزیع جوینت را بدست آوریم  و سپس مارجینالایز بکنیم و خود احتمال را بدست اوریم.

\begin{equation}
\begin{split}
P(T_2 > t) &= P(T_2 >t | T_1 ) f_{T_1}(k) \\ 
&= 
\int_0^{\infty} \frac{k+1}{k+t+1} 
\frac{1}{(1+k)^2}dk 
\\
&= \int_0^{\infty} \frac{1}{k+t+1} 
\frac{1}{1+k}dk 
\\
&= \frac{ln(1+t)}{t}
\end{split}
\end{equation}

اکثر راه را رفتیم پس توزیع 
CDF 
برابر است با:

\begin{equation}
F_{T_2}(t) = 1-P(T_2>t) = 1-\frac{ln(1+t)}{t}
\end{equation}
فقط یگ مشتق مانده که بگیریم و تمام شود. 

\begin{equation}
f_{T_2}(t) = \frac{\partial}{\partial t} F_{T_2}(t) = \frac{\partial}{\partial t} [1-\frac{ln(1+t)}{t}] = \dfrac{\ln\left(t+1\right)}{t^2}-\dfrac{1}{t\left(t+1\right)}
\end{equation}

\section{سوال ۳}
\subsection{a}
خب اول میریم سراغ تعریف های ریاضی و فهم بهتر مسئله. پس داریم:
\begin{gather}
N \sim Pois(\lambda) \\
S | N \sim Binomial(N,p)\\
T | N \sim Binomial(N,1-p)
\end{gather}

توزیع N که مشخص است. ابتدا توزیع S را دقیق بدست آوریم. (البته در ابتدا فرض کنیم n را میدانیم!) پس داریم:


\begin{equation}
P(S = s| N=n) = {n \choose s} p^{s} (1-p)^{n-s}
\end{equation}

حال میتوان توزیع S را با قانون احتمال کل بدست آورد!

\begin{equation}
\begin{split}
P(S=s) &= \sum_{n=s}^{\infty} P(S=s|N=n)P(N=n) 
\\ 
&= \sum_{n=s}^{\infty} {n \choose s} p^{s} (1-p)^{n-s}
\frac{e^{-\lambda} \lambda^n}{n!} 
\\
&= e^{-\lambda} p^s
\sum_{n=s}^{\infty} \frac{n!}{(n-s)! s!}
\frac{(1-p)^{n-s} \lambda^n}{n!} 
\\ 
&= \frac{ e^{-\lambda} p^s}{s!} 
\sum_{n=s}^{\infty} \frac{(1-p)^{n-s} \lambda^n}{(n-s)! } 
\\
&= \frac{ e^{-\lambda} p^s}{s!} 
\sum_{n=0}^{\infty} \frac{(1-p)^{n} \lambda^{n+s}}{n!} 
\\
&= \frac{ e^{-\lambda} \lambda^s p^s e^{(1-p)\lambda}}{s!} 
\\
&= 
\frac{e^{-p \lambda}}{(p \lambda)^s}{s!}\\
& \sim Pois(p \lambda) 
\end{split}
\end{equation}
برای حل بالا از بسط تیلور یا سری توانی استفاده شده است که به شکل زیر است:
\begin{equation}
e^x = \sum_{k=0}^{\infty} \frac{x^k}{k!}
\end{equation}
به طرز مشابه میتوان نشان داد که 
$T \sim Pois((1-p)\lambda)$
\subsection{b}
برای نشان داده این که دو متغیر از نظر آماری مستقل هستند باید نشان دهیم که توزیع جوینت آنها معادل ضرب توزیع حاشیه ای آنها است. پس داریم:

\begin{equation}
\begin{split}
f_{S,T}(s,t) &= P(S=s , T=t) 
\\
&= P(S=s , N = t+s ) 
\\
&= P(S=s | N=t+s)P(N=t+s) 
\\
&= 
{t+s \choose s} p^{s} (1-p)^{t}  
\frac{e^{-\lambda}  (\lambda)^{t+s}  }{(t+s)!}
\\
&= 
\frac{(t+s)!}{t! s!} p^s (1-p)^{t} 
e^{-\lambda} \lambda^t \lambda^s 
\frac{1}{(t+s)!}
\\
&= 
\frac{e^{-\lambda(1-p+p)} p^s \lambda^s (1-p)^{t} \lambda^t  }{t! s!}
\\ 
&= 
\frac{e^{-\lambda (1-p )} ((1-p)\lambda)^{t}}{t!}
\frac{e^{-\lambda p} (p\lambda)^s}{s!}
\\
&=
P(T=t) P(S=s) = Solved!
\end{split}
\end{equation}
\subsection{c}
خب فرض میکنیم متغیر تصادفی N
یک عدد ثابت مثل n باشد.
پس داریم:

\begin{equation}
\begin{split}
f_{S,T}(s,t) = P(S=s,T=t) 
&= 
P(S=s,N=s+t) \\&= P(N=s+t)P(S=s|N=s+t) = P(S=s)
\end{split}
\end{equation}

پس در این حالت می توان گفت هر کدام از S
و 
T
به یکدیگر کاملا وابسته هستند. 
\end{document}